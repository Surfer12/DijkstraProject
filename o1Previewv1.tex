\documentclass{article}
\usepackage{graphicx}
\usepackage{listings}
\usepackage{hyperref}
\usepackage{tikz}
\usetikzlibrary{graphs,graphdrawing}
\usegdlibrary{force}

\title{Comprehensive Research on Dijkstra’s Algorithm and Java Development Tools}
\author{}
\date{}

\begin{document}

\maketitle

\section{Introduction}

Dijkstra’s algorithm is a powerful method to find the shortest path between nodes in a graph. This document explores its implementation in Java, analyzes practical examples, and incorporates lightweight Java build tools like Gradle and IDE extensions for enhanced efficiency.

\section{Algorithm Overview}

\subsection{Definition}

Dijkstra’s algorithm calculates the shortest path in a weighted graph. It prioritizes exploration using the minimum cumulative distance at each step.

\subsection{Core Components}

\begin{itemize}
    \item \textbf{Priority Queue}: Maintains order based on the shortest known distance.
    \item \textbf{Edge Relaxation}: Updates tentative distances and paths.
    \item \textbf{Path Reconstruction}: Traces the optimal path from the destination to the source.
\end{itemize}

\section{Lightweight Java Development Workflow}

\subsection{Gradle: A Lightweight Build Tool}

\textbf{Advantages}:
\begin{itemize}
    \item Faster incremental builds compared to Maven.
    \item Declarative syntax with Kotlin/Groovy.
    \item Ideal for managing multi-module projects.
\end{itemize}

\textbf{Configuration}:

\begin{lstlisting}[language=Groovy, caption=Gradle Build Script]
plugins {
    id 'java'
    id 'application'
}

repositories {
    mavenCentral()
}

dependencies {
    implementation 'com.google.guava:guava:31.1-jre'
    testImplementation 'org.junit.jupiter:junit-jupiter:5.9.0'
}
\end{lstlisting}

\subsection{IDE and Extensions}

\begin{enumerate}
    \item \textbf{VS Code}:
    \begin{itemize}
        \item Extensions: Red Hat Java Tools, Gradle for Java, Debugger for Java.
        \item Lightweight and suited for smaller projects.
    \end{itemize}
    \item \textbf{IntelliJ IDEA}:
    \begin{itemize}
        \item Advanced debugging and deep Gradle integration.
        \item Ideal for large-scale projects.
    \end{itemize}
    \item \textbf{Oracle JDK}:
    \begin{itemize}
        \item Ensures compatibility with enterprise-grade systems.
    \end{itemize}
\end{enumerate}

\section{Dijkstra’s Algorithm: Practical Example}

\subsection{Graph Representation}

\begin{lstlisting}[language=Java, caption=Graph Representation in Java]
Map<Integer, Map<Integer, Integer>> graph = new HashMap<>();
graph.put(0, Map.of(1, 4, 6, 7));
graph.put(1, Map.of(0, 4, 2, 9, 6, 11));
graph.put(2, Map.of(1, 9, 3, 6, 4, 2));
// Additional nodes and weights...
\end{lstlisting}

\subsection{Algorithm Execution}

\begin{lstlisting}[language=Java, caption=Dijkstra's Algorithm Implementation]
public static Map<Integer, Integer> dijkstra(Map<Integer, Map<Integer, Integer>> graph, int start) {
    PriorityQueue<Node> pq = new PriorityQueue<>(Comparator.comparingInt(n -> n.weight));
    Map<Integer, Integer> distances = new HashMap<>();
    Map<Integer, Integer> previous = new HashMap<>();
    // Initialize distances and queue
    // Relax edges and update priority queue
    return distances;
}
\end{lstlisting}

\subsection{Path Reconstruction}

\begin{lstlisting}[language=Java, caption=Path Reconstruction Method]
private List<Integer> reconstructPath(Map<Integer, Integer> previous, int destination) {
    List<Integer> path = new ArrayList<>();
    for (Integer at = destination; at != null; at = previous.get(at)) {
        path.add(0, at);
    }
    return path;
}
\end{lstlisting}

\section{Examples and Analysis}

\subsection{Example Graph 1}

\begin{center}
\begin{tikzpicture}
    \graph [spring layout, nodes={circle, draw, inner sep=2pt}] {
        0 --[edge label={7}] 1;
        0 --[edge label={9}] 2;
        0 --[edge label={14}] 5;
        1 --[edge label={10}] 2;
        1 --[edge label={15}] 3;
        2 --[edge label={2}] 5;
    };
\end{tikzpicture}
\end{center}

\begin{itemize}
    \item \textbf{Shortest Path}: 0 → 2 → 5
    \item \textbf{Total Distance}: 11 units.
\end{itemize}

\subsection{Example Graph 2}

\begin{center}
\begin{tikzpicture}
    \graph [spring layout, nodes={circle, draw, inner sep=2pt}] {
        0 --[edge label={4}] 1;
        0 --[edge label={7}] 6;
        6 --[edge label={1}] 7;
        7 --[edge label={3}] 8;
    };
\end{tikzpicture}
\end{center}

\begin{itemize}
    \item \textbf{Shortest Path}: 0 → 6 → 7 → 8
    \item \textbf{Total Distance}: 11 units.
\end{itemize}

\subsection{Algorithm Steps}

\begin{enumerate}
    \item Initialize distances.
    \item Process nodes using priority queue.
    \item Relax edges and update neighbors.
    \item Reconstruct shortest paths.
\end{enumerate}

\section{Performance and Optimization}

\subsection{Time Complexity}

\begin{itemize}
    \item \textbf{Binary Heap}: $O((V + E) \log V)$
    \item \textbf{Fibonacci Heap}: $O(E + V \log V)$
\end{itemize}

\subsection{Optimizations}

\begin{enumerate}
    \item \textbf{Early Termination}:

\begin{lstlisting}[language=Java]
if (current.equals(destination)) break;
\end{lstlisting}

    \item \textbf{Priority Queue Updates}:
    \begin{itemize}
        \item Avoid redundant updates using decrease-key operations.
    \end{itemize}
    \item \textbf{Memory Management}:
    \begin{itemize}
        \item Use primitive arrays for small, dense graphs.
    \end{itemize}
\end{enumerate}

\section{Integration with Modern Java Tooling}

\subsection{Gradle with Red Hat Extensions}

\begin{itemize}
    \item Run builds, tests, and tasks efficiently within IDEs.
    \item Use Red Hat’s debugging and code analysis tools.
\end{itemize}

\subsection{IntelliJ IDEA Integration}

\begin{itemize}
    \item Automate project imports with Gradle or Maven.
    \item Debug complex graph-based applications seamlessly.
\end{itemize}

\subsection{Oracle Java Tools}

\begin{itemize}
    \item Ensure compliance with enterprise environments.
    \item Use NetBeans or IntelliJ for Java SE/EE projects.
\end{itemize}

\section{Conclusion}

Dijkstra’s algorithm, combined with lightweight build tools and advanced IDE extensions, forms a robust foundation for solving shortest path problems in various domains. Using Gradle and tools from Red Hat or Oracle enhances developer productivity, making the implementation scalable and efficient.

\end{document}